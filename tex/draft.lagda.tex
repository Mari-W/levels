\documentclass[runningheads,fleqn]{llncs}

\usepackage{fontspec}
\usepackage{unicode-math}
\usepackage[Latin,Greek]{ucharclasses}
\usepackage{amsmath}
\usepackage{stmaryrd}
\usepackage{newunicodechar}
\usepackage{proof}
\setlength{\inferLineSkip}{3pt}
\usepackage[backend=biber]{biblatex}
\addbibresource{references.bib}
\usepackage{tikz}
\usetikzlibrary{cd}
\usepackage{adjustbox}
\usepackage{syntax}

\title{More Universes (working title)}
\subtitle{Extend Agda's Universe Hierachy to $ε₀$}
\institute{Chair of Programming Languages, University of Freiburg \\
  \email{weidner@cs.uni-freiburg.de}}
\author{Marius Weidner}

\begin{document}

\maketitle

\begin{abstract}
\end{abstract}

% \section{Ordinals}
% 
% We mutual recursively define ordinals in cantor normal form (CNF) and a little above the limit $ε₀$:
% 
% \begin{grammar}
% <$o$> ::= 0
% \alt $ω ↑$ <$o₁$> + <$o₂$> \quad if $ε₀ > o₁ ≥$ exp($o₂$)
% \alt $ε₀+i$ \quad \quad \quad \quad \ \ \, $∀i ∈ ℕ$
% \end{grammar}
% \noindent and retrieving the exponent of an ordinal in cantor normal form (i.e. $o < ε₀$)
% \[
%   \text{exp}(o) =
%   \begin{cases}
%     0 & \text{if } o = 0 \\
%     o₁ & \text{if } o = ω ↑ o₁ + o₂
%   \end{cases}
% \]
% \noindent and ordering relation on oridnals $(o₁ , o₂) ∈ \_<\_$ where
% \[
%   o₁ < o₂ =
%   \begin{cases}
%     true & \text{if } o₁ = 0 \text{ and } o₂ = ω ↑ o₃ + o₄ \\
%     true & \text{if } o₁ = ω ↑ o₃ + o₄ \text{ and } o₂ = ω ↑ o₅ + o₆ \text{ and } o₃ < o₅ \\
%     true & \text{if } o₁ = ω ↑ o₃ + o₄ \text{ and } o₂ = ω ↑ o₅ + o₆ \text{ and } o₃ = o₅ \text{ and } o₄ < o₆ \\
%     true & \text{if } o₁ = ε₀ + i₁ \text{ and } o₂ ≠ ε₀ + i₂ \\
%     true & \text{if } o₁ = ε₀ + i₁ \text{ and } o₂ = ε₀ + i₂ \text{ and } i₁ <_ℕ i₂ \\
%     false & \text{otherwise}
%   \end{cases}
% \]
% and derive $≤$, $>$ and $≥$ accordingly
% \begin{align*}
%   o₁ ≤ o₂ &⇔ o₁ < o₂ \text{ or } o₁ = o₂ \\
%   o₁ > o₂ &⇔  o₂ < o₁ \\
%   o₁ ≥ o₂ &⇔ o₁ > o₂ \text{ or } o₁ = o₂ \\
% \end{align*}
% We also define the limit predicate and use $λ$ for limit oridnals
% \[
%   \text{lim}(o) =
%   \begin{cases}
%     true & \text{if } o = ω ↑ o₁ + 0  \\
%     true & \text{if } o = ω ↑ o₁ + o₂ \and{ and } o₁ ≠ 0 \text{ and lim} (o₂)} \\
%     true & \text{if } o = ε₀ + 0  \\
%     false & \text{otherwise}
%   \end{cases}
% \]
% \noindent and the least upper bound
% \[
%   o₁ ⊔ o₂ = ...
% \]

\section{Terms}

\subsection{Syntax}

\begin{grammar}
t ::= 
\alt x
\alt $λ$(x : $t₁$) $→ t₂$
\alt $t₁ \ t₂$
\alt $∀(x : t₁) → t₂$
\alt $t₁ ≡ t₂$
\alt refl
\alt $0$
\alt $ω ↑ t₁ +_{\{t\}} t₂$
\alt $λ^ℓ$(x : $t₁$) $→ t₂$
\alt $t₁ _{\{t\}} t₂$
\alt $∀^ℓ(ℓ : t₁) → t₂$
\alt suc $t$
\alt $t₁ ⊔ t₂$
\alt Set[$t$]
\alt Set[$ε₀+i$] for all $i ∈ ℕ$
\alt Level
\end{grammar}

\subsection{Typing}

% \infer[\text{T-OrdZero}]{
%   Γ ⊢ 0 : Ord
% }{}
% \bigskip
% \infer[\text{T-OrdCnf}]{
%   Γ ⊢ ω ↑ o₁ + o₂ : Ord
% }{
%   Γ ⊢ o₁ : Ord &
%   Γ ⊢ o₂ : Ord &
%   exp(o₂) < o₁
% }
% \bigskip
% \infer[\text{T-OrdEps}]{
%   Γ ⊢ ε₀ + i : Ord
% }{}
% \bigskip
% \infer[\text{T-OrdSuc}]{
%   Γ ⊢ suc \ o : Lvl[λ]
% }{
%   Γ ⊢ o : Lvl[λ]
% }
% \bigskip
% \infer[\text{T-OrdLub}]{
%   Γ ⊢ o₁ ⊔ o₂ : Lvl[λ₁ ⊔ λ₂]
% }{
%   Γ ⊢ o₁ : Lvl[λ₁] &
%   Γ ⊢ o₂ : Lvl[λ₂] 
% }
% \bigskip
% % \infer[\text{T-SetOrd}]{
% %   Γ ⊢ Ord : Set[0]
% % }{}
% % \bigskip
% \infer[\text{T-LimOrd}]{
%   Γ ⊢ o : Lvl[λ]
% }{
%   \text{lim}(λ) ∧ o < λ
% }
% \bigskip
% \infer[\text{T-SetLim}]{
%   Γ ⊢ Lvl[λ] : Set[λ]
% }{
%   \text{lim}(λ)
% }
% \bigskip
% \infer[\text{T-SetSet}]{
%   Γ ⊢ Set[o] : Set[suc \ o]
% }{}
% \bigskip
% \bigskip
% \bigskip
% \bigskip
% \infer[\text{T-Var}]{
%   Γ ⊢ x : t
% }{
%   (x : t) ∈ Γ
% }
% \bigskip
% \infer[\text{T-Abs}]{
%   Γ ⊢ λ(x : t₁) → e : t₂
% }{
%   Γ ⊢ t₁ : Set[o₁] &
%   Γ, x : t₁ ⊢ e : ∀(x : t₁) → t₂
% }
% \bigskip
% \infer[\text{T-App}]{
%   Γ ⊢ e₁ e₂ : t₂[x/e₂]
% }{
%   Γ ⊢ e₁ : ∀(x : t₁) → t₂
%   Γ ⊢ e₂ : t₁ &
% }
% \bigskip
% \infer[\text{T-All}]{
%   Γ ⊢ ∀(x: t₁) → t₁ : Set[o₁ ⊔ suc \ o₂]
% }{
%   Γ ⊢ t₁ : Set[o₁] &
%   Γ, x: t₁ ⊢ t₂ : Set[o₂]
% }
% \bigskip
% \infer[\text{T-Conv}]{
%   Γ ⊢ e : t₂
% }{
%   Γ ⊢ e : t₁ &
%   t₁ ≡ t₂
% }
\printbibliography{}

\end{document}