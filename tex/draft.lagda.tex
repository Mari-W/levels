\documentclass[runningheads,fleqn]{llncs}

\usepackage{fontspec}
\usepackage{unicode-math}
\usepackage[Latin,Greek]{ucharclasses}
\usepackage{amsmath}
\usepackage{stmaryrd}
\usepackage{newunicodechar}
\usepackage{proof}
\setlength{\inferLineSkip}{3pt}
\usepackage[backend=biber]{biblatex}
\addbibresource{references.bib}
\usepackage{tikz}
\usetikzlibrary{cd}
\usepackage{adjustbox}

\title{Taking Control Over The Multiverse}
\institute{Chair of Programming Languages, University of Freiburg \\
  \email{weidner@cs.uni-freiburg.de}}
\author{Marius Weidner}

\begin{document}

\maketitle

\begin{abstract}
\end{abstract}

\subsection{Syntax}

\begin{align*}
  t, ℓ, A, B ::&= x \\
  &|\ λ(x : A) → t \\
  &|\ t₁ \ t₂ \\
  &|\ ∀(x : A) → B \\
  &|\ t₁ ≡_A t₂ \\
  &|\ \texttt{refl} \ t \\
  &|\ A ⊎ B \\ 
  &|\ \texttt{inj}₁ \ t \\ 
  &|\ \texttt{inj}₂ \ t \\ 
  &|\ \texttt{case } t \texttt{ of } \\
  & \quad \texttt{inj}₁ \ t → t₁ \\
  & \quad \texttt{inj}₂ \ t → t₂ \\ 
  &|\ ⊥ \\
  &|\ \texttt{Level} \\
  &|\ 0 \\
  &|\ ω ↑ ℓ₁ +_t ℓ₂ & ∷ (ℓ₁ : \texttt{Level}) → (ℓ₂ : \texttŧ{Level}) → (t : ↑ ℓ₂ ≤ ℓ₁) \\
  &|\ \texttt{case}_ℓ \ t \texttt{ of } & \texttt{\{-\# OPTIONS --undecidable-type-checking \#-\}} \\
  & \quad 0→ t₁ \\
  & \quad ω ↑ ℓ₁ +_t ℓ₂→ t₂ \\
  &|\ \texttt{suc } ℓ \\
  &|\ ↑ ℓ \\
  &|\ ℓ₁ ⊔ ℓ₂ \\
  &|\ ℓ₁ <_ℓ ℓ₂ \ & \texttt{\{-\# OPTIONS --undecidable-type-checking \#-\}} \\ 
  &|\ <_{ℓ₁} & ∷ 0 <_ℓ ω ↑ ℓ₁ +_t ℓ₂\\
  &|\ <_{ℓ₂} \ t & ∷ ℓ₁₁ <_ℓ ℓ₂₁ → (ω ↑ ℓ₁₁ +_t ℓ₁₂) <_ℓ (ω ↑ ℓ₂₁ +_t ℓ₂₂)\\
  &|\ <_{ℓ₃} \ t \ t′ & ∷ ℓ₁₁ ≡ ℓ₂₁ → ℓ₂₁ <_ℓ ℓ₂₂ → (ω ↑ ℓ₁₁ +_t ℓ₁₂) <_ℓ (ω ↑ ℓ₂₁ +_t ℓ₂₂) \\
  &|\ \texttt{case}_{<_ℓ} \ t \texttt{ of } & \texttt{\{-\# OPTIONS --undecidable-type-checking \#-\}} \\
  & \quad <_{ℓ₁}→ t₁ \\
  & \quad <_{ℓ₂} \ t → t₂ \\
  & \quad <_{ℓ₂} \ t \ t′ → t₃ \\
  &|\ \texttt{Level}_ℓ \\
  &|\ ℓ ,_ℓ t \\  
  &|\ \texttt{proj}_ℓ \ t \\
  &|\ \texttt{proj}_{<_ℓ} \ t \\
  &|\ \texttt{Set}_ℓ \\
  &|\ \texttt{Set}_{ε_0+i} & \texttt{ for all } i ∈ ℕ \\
\end{align*}

We write $\texttt{Set}$ for $\texttt{Set}_0$.

We also write $ℓ₁ ≤_ℓ ℓ₂$ as shorthand for $ℓ₁ <_ℓ ℓ₂ ⊎ ℓ₁ ≡ ℓ₂$.

% We define $↑ ℓ$ for retrieving the exponent of an ordinal in CNF $\texttt{case}_ℓ \ t \texttt{ of } 0→ 0; ω ↑ ℓ₁ + ℓ₂→ ℓ₁$

We might omit the proof $t$ that $↑ ℓ₂ ≤ ℓ₁$ in the constructor $ω ↑ ℓ₁ +_t ℓ₂$ if it follows from context.

By an abuse of notation we may write $\texttt{f} : ∀(ℓ : \texttt{Level}_{ℓ′}) → \texttt{Set } ℓ$ and $\texttt{f} \ ℓ \ \{ℓ<ℓ′\}$ instead of $\texttt{f} : ∀(ℓ : \texttt{Level}_{ℓ′}) → \texttt{Set } (\texttt{proj}_ℓ \ ℓ)$ and $\texttt{f } (ℓ ,_ℓ ℓ<ℓ′)$ which is closer to what we believe should be implemented.

% We define $\texttt{suc } ℓ$ as the successor of any ordinals in CNF $\texttt{case}_ℓ \ t \texttt{ of } 0→ ω ↑ 0 + 0; ω ↑ ℓ₁ + ℓ₂→ ω ↑ ℓ₁ + \texttt{suc } ℓ₂$.

Note that $\texttt{suc } ℓ$ and $↑ ℓ$ are essentially just definitions possible when \texttt{\{-\# OPTIONS --undecidable-type-checking \#-\}} is enabled.
We could implement them in an manually checked unsafe module and mark them for the compiler similar to the constructors of Level.

\begin{itemize}
 \item All syntax constructs marked with \texttt{\{-\# OPTIONS --undecidable-type-checking \#-\}} should only be visible to the compiler / some manually checked prelude module that included the least definitions introduced above. 
 \item Note that in the case of level quantification the user \emph{sees} $\texttt{\_}<_ℓ\texttt{\_}$ \emph{indirectly} in a secure way.
 \item IDEA: Can we allow $\texttt{\_}<_ℓ\texttt{\_}$ to appear in the \emph{return type} of a function without breaking decidability of typechecking? Further: We could allow \emph{fully generalized} $ℓ₁ <_ℓ ℓ₂$ as argument.
 \item Enabling the option for use in any other module, enables the user to break decidability of typechecking but also allows to add custom laws. 
\end{itemize}

\subsection{Laws}
Idempotence: $ℓ ⊔ ℓ ≡ ℓ$ \\ 
Associativity: $(ℓ₁ ⊔ ℓ₂) ⊔ ℓ₃ ≡ ℓ₁ ⊔ (ℓ₂ ⊔ ℓ₃)$ \\
Commutativity: $ℓ₁ ⊔ ℓ₂ ≡ ℓ₂ ⊔ ℓ₁$ \\
Distributivity$₁$: $\text{suc } (ℓ₁ ⊔ ℓ₂) ≡ \texttt{suc } ℓ₁ ⊔ \texttt{suc } ℓ₂$ \\
Distributivity$₂$: $ω ↑ ℓ +_{t} (ℓ₁ ⊔ ℓ₂) ≡ ω ↑ ℓ +_{t₁} ℓ₁ ⊔ ω ↑ ℓ +_{t₂} ℓ₂$ \\
Distributivity$₃$: $↑ (ℓ₁ ⊔ ℓ₂) ≡ ↑ ℓ₁ ⊔ ↑ ℓ₂$ \\
Neutrality: $ℓ ⊔ 0 ≡ ℓ$ \\
Subsumption$₁$: $ℓ ⊔ \texttt{suc}ⁿ \ ℓ ≡ \texttt{suc}ⁿ \ ℓ$\\
Subsumption$₂$: $ℓ ⊔ ω ↑ ℓ₁ + .. + ω ↑ ℓₙ + \texttt{suc}ⁿ \ ℓ ≡ ω ↑ ℓ₁ + .. + ω ↑ ℓₙ + \texttt{suc}ⁿ \ ℓ$\\
Subsumption$₃$: $ℓ ⊔ ↑ⁿ ℓ ≡ ℓ$ 

\begin{itemize}
  \item All laws should be provable when \texttt{\{-\# OPTIONS --undecidable-type-checking \#-\}} is enabled
  \item With  \texttt{\{-\# OPTIONS --undecidable-type-checking \#-\}} enabled you can add more reduction rules (either applying them explicitly or by using \texttt{\{-\# REWRITE \#-\}} 
  \item The rewrite system is `best effort', i.e. \emph{not} complete..
  \item .. it might be confluent though (it probably even \emph{needs} to be?)
\end{itemize}

We should probably also include a library of manually checked equations enabling inequality-reasoning that make use of \texttt{\{-\# OPTIONS --undecidable-type-checking \#-\}}

\noindent Transitivity: $ℓ₁ <_ℓ ℓ₂ → ℓ₂ <_ℓ ℓ₃ → ℓ₁ <_ℓ ℓ₃$ \\ 
Subsumption: $ℓ₁ <_ℓ ℓ₂ → ℓ₁ <_ℓ ℓ₂ ⊔ ℓ₃$

\subsection{Typing}

\infer[\text{T-Var}]{
  Γ ⊢ x : T
}{
  (x : T) ∈ Γ 
}
\noindent todo: add context well formedness(?) 

\bigskip
\infer[\text{T-Abs}]{
  Γ ⊢ λ(x : A) → t : ∀(x : A) → B
}{
  Γ, x : A ⊢ t : B &
  Γ ⊢ A : \texttt{Set}_ℓ
}
\bigskip
\infer[\text{T-App}]{
  Γ ⊢ t₁ \ t₂ : B[x/t₂]
}{
  Γ ⊢ t₁ : ∀(x : A) → B &
  Γ ⊢ t₂ : A
}
\bigskip
\infer[\text{T-All}]{
  Γ ⊢ ∀(x: A) → B : \texttt{Set}_{ℓ₁ ⊔ ℓ₂}
}{
  Γ ⊢ A : \texttt{Set}_{ℓ₁} &
  Γ, x : A ⊢ B : \texttt{Set}_{ℓ₂}
}
\bigskip
\infer[\text{T-Conv}]{
  Γ ⊢ t₁ : A₁
}{
  Γ ⊢ t₂ : A₂ &
  Γ ⊢ A₁ = A₂ : \texttt{Set}_ℓ
}
\noindent todo: add definitional equality rules(?)

\bigskip
\infer[\text{T-Set}]{
  Γ ⊢ \texttt{Set}_ℓ : \texttt{Set}_{\texttt{suc } ℓ}
}{
  Γ ⊢ ℓ : \texttt{Level}
}
\noindent todo: add context well formedness(?) 

\bigskip
\infer[\text{T-Eq}]{
  Γ ⊢ t₁ ≡_A t₂: \texttt{Set}_ℓ
}{
  Γ ⊢ t₁ : A &
  Γ ⊢ t₂ : A &
  Γ ⊢ A : \texttt{Set}_ℓ
}
\noindent $A ⊎ B$, $\texttt{inj}₁$, $\texttt{inj}₂$, \texttt{case\_of\_} missing

\bigskip
\infer[\text{T-Level}]{
  Γ ⊢ \texttt{Level} : \texttt{Set}_{ε₀}
}{
}
\bigskip
\infer[\text{T-Zero}]{
  Γ ⊢ 0 : \texttt{Level}
}{
}
\bigskip
\infer[\text{T-CNF}]{
  Γ ⊢ ω ↑ ℓ₁ +_t ℓ₂ : \texttt{Level}
}{
  Γ ⊢ ℓ₁ : \texttt{Level} &
  Γ ⊢ ℓ₂ : \texttt{Level} &
  Γ ⊢ t : ↑ ℓ₂ ≤_ℓ ℓ₁
}
\noindent $\texttt{case}_ℓ\texttt{\_of\_}}$ missing

\bigskip
\infer[\text{T-Suc}]{
  Γ ⊢ \texttt {suc } ℓ : \texttt{Level}
}{
  Γ ⊢ ℓ : \texttt{Level} 
}
\bigskip
\infer[\text{T-Exp}]{
  Γ ⊢ ↑ ℓ : \texttt{Level}
}{
  Γ ⊢ ℓ : \texttt{Level} 
}
\bigskip
\infer[\text{T-LUB}]{
  Γ ⊢ ℓ₁ ⊔ ℓ₂ : \texttt{Level}
}{
  Γ ⊢ ℓ₁ : \texttt{Level} &
  Γ ⊢ ℓ₂ : \texttt{Level} 
}
\bigskip
\infer[\text{T-LT}]{
  Γ ⊢ ℓ₁ <_ℓ ℓ₂ : \texttt{Set}
}{
  Γ ⊢ ℓ₁ : \texttt{Level} &
  Γ ⊢ ℓ₂ : \texttt{Level} 
}
\bigskip
\infer[\text{T-LTZero}]{
  Γ ⊢ <_{ℓ_1} : 0 <_ℓ  ω ↑ ℓ₁ +_t ℓ₂
}{
  Γ ⊢ ℓ₁ : \texttt{Level} &
  Γ ⊢ ℓ₂ : \texttt{Level} &
  Γ ⊢ t : ↑ ℓ₂ ≤_ℓ ℓ₁
}
\bigskip
\infer[\text{T-LTExp}]{
  Γ ⊢ <_{ℓ_1} t : ω ↑ ℓ₁ +_t₁ ℓ₂ <_ℓ  ω ↑ ℓ₃ +_t₂ ℓ₄
}{
  Γ ⊢ ℓ_{1..4} : \texttt{Level} &
  Γ ⊢ t₁ : ↑ ℓ₂ ≤_ℓ ℓ₁ &
  Γ ⊢ t₂ : ↑ ℓ₄ ≤_ℓ ℓ₃ &
  Γ ⊢ t : ℓ₁ <_ℓ ℓ₃
}
\bigskip
\infer[\text{T-LTCons}]{
  Γ ⊢ <_{ℓ_1} t t′ : ω ↑ ℓ₁ +_t₁ ℓ₂ <_ℓ  ω ↑ ℓ₃ +_t₂ ℓ₄
}{
  Γ ⊢ ℓ_{1..4} : \texttt{Level} &
  Γ ⊢ t₁ : ↑ ℓ₂ ≤_ℓ ℓ₁ &
  Γ ⊢ t₂ : ↑ ℓ₄ ≤_ℓ ℓ₃ &
  Γ ⊢ t : ℓ₁ ≡ ℓ₃ &
  Γ ⊢ t′ : ℓ₂ <_ℓ ℓ₄
}
\noindent $\texttt{case}_{<_ℓ}\texttt{\_of\_}}$ missing

\bigskip
\infer[\text{T-BoundLevel}]{
  Γ ⊢ \texttt{Level}_ℓ : \texttt{Set } ℓ
}{
  Γ ⊢ ℓ : \texttt{Level}
}
\bigskip
\infer[\text{T-LevelPair}]{
  Γ ⊢ ℓ ,_ℓ t : \texttt{Level}_{ℓ′}
}{
  Γ ⊢ ℓ : \texttt{Level} &
  Γ ⊢ ℓ′ : \texttt{Level} &
  Γ ⊢ t : ℓ <_ℓ ℓ′
}
\bigskip
\infer[\text{T-LevelBoundProj}]{
  Γ ⊢ \texttt{proj}_ℓ \ t : \texttt{Level}
}{
  Γ ⊢ t : \texttt{Level}_ℓ &
}
\bigskip
\infer[\text{T-LevelBoundProofProj}]{
  Γ ⊢ \texttt{proj}_{<_ℓ} \ t : (\texttt{proj}_ℓ \ t) <_ℓ ℓ
}{
  Γ ⊢ t : \texttt{Level}_ℓ &
}
\bigskip
\infer[\text{T-Set}]{
  Γ ⊢ \texttt{Set}_ℓ : \texttt{Set}_{\texttt{suc } ℓ}
}{
  Γ ⊢ ℓ : \texttt{Level} &
}
\bigskip
\infer[\text{T-SetEps}]{
  Γ ⊢ \texttt{Set}_{ε_0 + i} : \texttt{Set}_{ε_0 + i + 1} \texttt{ for all } i ∈ ℕ
}{
}

\subsection{Semantics}
\bigskip
\infer[\text{$β$-suc-$0$}]{
  \texttt{suc } 0 ↪ ω ↑ 0 + 0
}{
}
\bigskip
\infer[\text{$β$-suc-$ω$}]{
  \texttt{suc } ω ↑ ℓ₁ + ℓ₂ ↪ ω ↑ ℓ₁ + \texttt{suc } ℓ₂
}{
}
\bigskip
\infer[\text{$β$-$↑$-$0$}]{
  ↑ 0 ↪ 0
}{
}
\bigskip
\infer[\text{$β$-$↑$-$ω$}]{
  ↑ (ω ↑ ℓ₁ + ℓ₂) ↪ ℓ₁
}{
}
\bigskip

\subsection{Metatheory}

\begin{theorem}[Intrinsic Level Properties]
The laws from Section 0.2 are correct, i.e. can be proven by induction when \texttt{\{-\# OPTIONS --undecidable-type-checking \#-\}} is enabled.
Furthermore the resulting rewrite system is confluent, i.e. satisfies the diamond property. 
\end{theorem}

\begin{theorem}[Soundness]
The system is sound, i.e. progress and subject reduction hold. 
Progress hold if we have $∅ ⊢ t : A$ then either $t$ is in weak head normal form or $∃ t'. t ↪ t'$. 
Subject reduction holds if reduction preserves typing, i.e. if $Γ ⊢ t : A$ and $t ↪ t'$ then $Γ ⊢ t' : A$.
\end{theorem}

\begin{theorem}[Logical Consistency]
The system is logical consistent if $∅ ⊢ t : ⊥$ is not derivable. 
This proof requires an logical relation.
This should hold even when \texttt{\{-\# OPTIONS --undecidable-type-checking \#-\}} is enabled.
\end{theorem}

\begin{theorem}[Decidability of Type Checking]
With \texttt{\{-\# OPTIONS --undecidable-type-checking \#-\}} disabled (and no usage of \texttt{\{-\# TERMINATING \#-\}} or similar) the type checking procedure terminates. When enabled, type checking may run forever. 
\end{theorem}

\printbibliography{}

\end{document}